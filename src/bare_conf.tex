
%% bare_conf.tex
%% V1.4b
%% 2015/08/26
%% by Michael Shell
%% See:
%% http://www.michaelshell.org/
%% for current contact information.
%%
%% This is a skeleton file demonstrating the use of IEEEtran.cls
%% (requires IEEEtran.cls version 1.8b or later) with an IEEE
%% conference paper.
%%
%% Support sites:
%% http://www.michaelshell.org/tex/ieeetran/
%% http://www.ctan.org/pkg/ieeetran
%% and
%% http://www.ieee.org/

%%*************************************************************************
%% Legal Notice:
%% This code is offered as-is without any warranty either expressed or
%% implied; without even the implied warranty of MERCHANTABILITY or
%% FITNESS FOR A PARTICULAR PURPOSE!
%% User assumes all risk.
%% In no event shall the IEEE or any contributor to this code be liable for
%% any damages or losses, including, but not limited to, incidental,
%% consequential, or any other damages, resulting from the use or misuse
%% of any information contained here.
%%
%% All comments are the opinions of their respective authors and are not
%% necessarily endorsed by the IEEE.
%%
%% This work is distributed under the LaTeX Project Public License (LPPL)
%% ( http://www.latex-project.org/ ) version 1.3, and may be freely used,
%% distributed and modified. A copy of the LPPL, version 1.3, is included
%% in the base LaTeX documentation of all distributions of LaTeX released
%% 2003/12/01 or later.
%% Retain all contribution notices and credits.
%% ** Modified files should be clearly indicated as such, including  **
%% ** renaming them and changing author support contact information. **
%%*************************************************************************


% *** Authors should verify (and, if needed, correct) their LaTeX system  ***
% *** with the testflow diagnostic prior to trusting their LaTeX platform ***
% *** with production work. The IEEE's font choices and paper sizes can   ***
% *** trigger bugs that do not appear when using other class files.       ***                          ***
% The testflow support page is at:
% http://www.michaelshell.org/tex/testflow/

\documentclass[conference]{IEEEtran}
% Some Computer Society conferences also require the compsoc mode option,
% but others use the standard conference format.
%
% If IEEEtran.cls has not been installed into the LaTeX system files,
% manually specify the path to it like:
% \documentclass[conference]{../sty/IEEEtran}

% \usepackage[brazil]{babel}
% \usepackage[latin1]{inputenc}

\usepackage{graphicx,url}
\usepackage{float}

\usepackage[brazil]{babel}
\usepackage[utf8]{inputenc}
\usepackage[T1]{fontenc}


% Some very useful LaTeX packages include:
% (uncomment the ones you want to load)


% *** MISC UTILITY PACKAGES ***
%
%\usepackage{ifpdf}
% Heiko Oberdiek's ifpdf.sty is very useful if you need conditional
% compilation based on whether the output is pdf or dvi.
% usage:
% \ifpdf
%   % pdf code
% \else
%   % dvi code
% \fi
% The latest version of ifpdf.sty can be obtained from:
% http://www.ctan.org/pkg/ifpdf
% Also, note that IEEEtran.cls V1.7 and later provides a builtin
% \ifCLASSINFOpdf conditional that works the same way.
% When switching from latex to pdflatex and vice-versa, the compiler may
% have to be run twice to clear warning/error messages.






% *** CITATION PACKAGES ***
%
%\usepackage{cite}
% cite.sty was written by Donald Arseneau
% V1.6 and later of IEEEtran pre-defines the format of the cite.sty package
% \cite{} output to follow that of the IEEE. Loading the cite package will
% result in citation numbers being automatically sorted and properly
% "compressed/ranged". e.g., [1], [9], [2], [7], [5], [6] without using
% cite.sty will become [1], [2], [5]--[7], [9] using cite.sty. cite.sty's
% \cite will automatically add leading space, if needed. Use cite.sty's
% noadjust option (cite.sty V3.8 and later) if you want to turn this off
% such as if a citation ever needs to be enclosed in parenthesis.
% cite.sty is already installed on most LaTeX systems. Be sure and use
% version 5.0 (2009-03-20) and later if using hyperref.sty.
% The latest version can be obtained at:
% http://www.ctan.org/pkg/cite
% The documentation is contained in the cite.sty file itself.






% *** GRAPHICS RELATED PACKAGES ***
%
\ifCLASSINFOpdf
  % \usepackage[pdftex]{graphicx}
  % declare the path(s) where your graphic files are
  % \graphicspath{{../pdf/}{../jpeg/}}
  % and their extensions so you won't have to specify these with
  % every instance of \includegraphics
  % \DeclareGraphicsExtensions{.pdf,.jpeg,.png}
\else
  % or other class option (dvipsone, dvipdf, if not using dvips). graphicx
  % will default to the driver specified in the system graphics.cfg if no
  % driver is specified.
  % \usepackage[dvips]{graphicx}
  % declare the path(s) where your graphic files are
  % \graphicspath{{../eps/}}
  % and their extensions so you won't have to specify these with
  % every instance of \includegraphics
  % \DeclareGraphicsExtensions{.eps}
\fi
% graphicx was written by David Carlisle and Sebastian Rahtz. It is
% required if you want graphics, photos, etc. graphicx.sty is already
% installed on most LaTeX systems. The latest version and documentation
% can be obtained at:
% http://www.ctan.org/pkg/graphicx
% Another good source of documentation is "Using Imported Graphics in
% LaTeX2e" by Keith Reckdahl which can be found at:
% http://www.ctan.org/pkg/epslatex
%
% latex, and pdflatex in dvi mode, support graphics in encapsulated
% postscript (.eps) format. pdflatex in pdf mode supports graphics
% in .pdf, .jpeg, .png and .mps (metapost) formats. Users should ensure
% that all non-photo figures use a vector format (.eps, .pdf, .mps) and
% not a bitmapped formats (.jpeg, .png). The IEEE frowns on bitmapped formats
% which can result in "jaggedy"/blurry rendering of lines and letters as
% well as large increases in file sizes.
%
% You can find documentation about the pdfTeX application at:
% http://www.tug.org/applications/pdftex





% *** MATH PACKAGES ***
%
%\usepackage{amsmath}
% A popular package from the American Mathematical Society that provides
% many useful and powerful commands for dealing with mathematics.
%
% Note that the amsmath package sets \interdisplaylinepenalty to 10000
% thus preventing page breaks from occurring within multiline equations. Use:
%\interdisplaylinepenalty=2500
% after loading amsmath to restore such page breaks as IEEEtran.cls normally
% does. amsmath.sty is already installed on most LaTeX systems. The latest
% version and documentation can be obtained at:
% http://www.ctan.org/pkg/amsmath





% *** SPECIALIZED LIST PACKAGES ***
%
%\usepackage{algorithmic}
% algorithmic.sty was written by Peter Williams and Rogerio Brito.
% This package provides an algorithmic environment fo describing algorithms.
% You can use the algorithmic environment in-text or within a figure
% environment to provide for a floating algorithm. Do NOT use the algorithm
% floating environment provided by algorithm.sty (by the same authors) or
% algorithm2e.sty (by Christophe Fiorio) as the IEEE does not use dedicated
% algorithm float types and packages that provide these will not provide
% correct IEEE style captions. The latest version and documentation of
% algorithmic.sty can be obtained at:
% http://www.ctan.org/pkg/algorithms
% Also of interest may be the (relatively newer and more customizable)
% algorithmicx.sty package by Szasz Janos:
% http://www.ctan.org/pkg/algorithmicx




% *** ALIGNMENT PACKAGES ***
%
%\usepackage{array}
% Frank Mittelbach's and David Carlisle's array.sty patches and improves
% the standard LaTeX2e array and tabular environments to provide better
% appearance and additional user controls. As the default LaTeX2e table
% generation code is lacking to the point of almost being broken with
% respect to the quality of the end results, all users are strongly
% advised to use an enhanced (at the very least that provided by array.sty)
% set of table tools. array.sty is already installed on most systems. The
% latest version and documentation can be obtained at:
% http://www.ctan.org/pkg/array


% IEEEtran contains the IEEEeqnarray family of commands that can be used to
% generate multiline equations as well as matrices, tables, etc., of high
% quality.




% *** SUBFIGURE PACKAGES ***
%\ifCLASSOPTIONcompsoc
%  \usepackage[caption=false,font=normalsize,labelfont=sf,textfont=sf]{subfig}
%\else
%  \usepackage[caption=false,font=footnotesize]{subfig}
%\fi
% subfig.sty, written by Steven Douglas Cochran, is the modern replacement
% for subfigure.sty, the latter of which is no longer maintained and is
% incompatible with some LaTeX packages including fixltx2e. However,
% subfig.sty requires and automatically loads Axel Sommerfeldt's caption.sty
% which will override IEEEtran.cls' handling of captions and this will result
% in non-IEEE style figure/table captions. To prevent this problem, be sure
% and invoke subfig.sty's "caption=false" package option (available since
% subfig.sty version 1.3, 2005/06/28) as this is will preserve IEEEtran.cls
% handling of captions.
% Note that the Computer Society format requires a larger sans serif font
% than the serif footnote size font used in traditional IEEE formatting
% and thus the need to invoke different subfig.sty package options depending
% on whether compsoc mode has been enabled.
%
% The latest version and documentation of subfig.sty can be obtained at:
% http://www.ctan.org/pkg/subfig




% *** FLOAT PACKAGES ***
%
%\usepackage{fixltx2e}
% fixltx2e, the successor to the earlier fix2col.sty, was written by
% Frank Mittelbach and David Carlisle. This package corrects a few problems
% in the LaTeX2e kernel, the most notable of which is that in current
% LaTeX2e releases, the ordering of single and double column floats is not
% guaranteed to be preserved. Thus, an unpatched LaTeX2e can allow a
% single column figure to be placed prior to an earlier double column
% figure.
% Be aware that LaTeX2e kernels dated 2015 and later have fixltx2e.sty's
% corrections already built into the system in which case a warning will
% be issued if an attempt is made to load fixltx2e.sty as it is no longer
% needed.
% The latest version and documentation can be found at:
% http://www.ctan.org/pkg/fixltx2e


%\usepackage{stfloats}
% stfloats.sty was written by Sigitas Tolusis. This package gives LaTeX2e
% the ability to do double column floats at the bottom of the page as well
% as the top. (e.g., "\begin{figure*}[!b]" is not normally possible in
% LaTeX2e). It also provides a command:
%\fnbelowfloat
% to enable the placement of footnotes below bottom floats (the standard
% LaTeX2e kernel puts them above bottom floats). This is an invasive package
% which rewrites many portions of the LaTeX2e float routines. It may not work
% with other packages that modify the LaTeX2e float routines. The latest
% version and documentation can be obtained at:
% http://www.ctan.org/pkg/stfloats
% Do not use the stfloats baselinefloat ability as the IEEE does not allow
% \baselineskip to stretch. Authors submitting work to the IEEE should note
% that the IEEE rarely uses double column equations and that authors should try
% to avoid such use. Do not be tempted to use the cuted.sty or midfloat.sty
% packages (also by Sigitas Tolusis) as the IEEE does not format its papers in
% such ways.
% Do not attempt to use stfloats with fixltx2e as they are incompatible.
% Instead, use Morten Hogholm'a dblfloatfix which combines the features
% of both fixltx2e and stfloats:
%
% \usepackage{dblfloatfix}
% The latest version can be found at:
% http://www.ctan.org/pkg/dblfloatfix




% *** PDF, URL AND HYPERLINK PACKAGES ***
%
%\usepackage{url}
% url.sty was written by Donald Arseneau. It provides better support for
% handling and breaking URLs. url.sty is already installed on most LaTeX
% systems. The latest version and documentation can be obtained at:
% http://www.ctan.org/pkg/url
% Basically, \url{my_url_here}.




% *** Do not adjust lengths that control margins, column widths, etc. ***
% *** Do not use packages that alter fonts (such as pslatex).         ***
% There should be no need to do such things with IEEEtran.cls V1.6 and later.
% (Unless specifically asked to do so by the journal or conference you plan
% to submit to, of course. )


% correct bad hyphenation here
\hyphenation{op-tical net-works semi-conduc-tor}


\begin{document}
%
% paper title
% Titles are generally capitalized except for words such as a, an, and, as,
% at, but, by, for, in, nor, of, on, or, the, to and up, which are usually
% not capitalized unless they are the first or last word of the title.
% Linebreaks \\ can be used within to get better formatting as desired.
% Do not put math or special symbols in the title.
\title{Ferramenta de Qualidade de Imagens 360 Integrada ao Editor Unity}


% author names and affiliations
% use a multiple column layout for up to three different
% affiliations
% \author{Adriano M. Gil\inst{1}, Eliamara Silva\inst{1}, Thiago S. Figueira\inst{1}}

% \address{Samsung Instituto de Desenvolvimento para a Informática da Amazônia
%   (SIDIA)\\
%   Manaus -- AM -- Brazil
%   \email{\{adriano.gil,eliamara.s,t.figueira\}@samsung.com}
% }

\author{\IEEEauthorblockN{Adriano M. Gil\IEEEauthorrefmark{1}
Eliamara Silva\IEEEauthorrefmark{1},
Thiago S. Figueira\IEEEauthorrefmark{1}}
\IEEEauthorblockA{\IEEEauthorrefmark{1}Samsung Instituto de Desenvolvimento para a Informática da Amazônia\\
SIDIA,\\
Manaus -- AM -- Brazil}}

% conference papers do not typically use \thanks and this command
% is locked out in conference mode. If really needed, such as for
% the acknowledgment of grants, issue a \IEEEoverridecommandlockouts
% after \documentclass

% for over three affiliations, or if they all won't fit within the width
% of the page, use this alternative format:
%
%\author{\IEEEauthorblockN{Michael Shell\IEEEauthorrefmark{1},
%Homer Simpson\IEEEauthorrefmark{2},
%James Kirk\IEEEauthorrefmark{3},
%Montgomery Scott\IEEEauthorrefmark{3} and
%Eldon Tyrell\IEEEauthorrefmark{4}}
%\IEEEauthorblockA{\IEEEauthorrefmark{1}School of Electrical and Computer Engineering\\
%Georgia Institute of Technology,
%Atlanta, Georgia 30332--0250\\ Email: see http://www.michaelshell.org/contact.html}
%\IEEEauthorblockA{\IEEEauthorrefmark{2}Twentieth Century Fox, Springfield, USA\\
%Email: homer@thesimpsons.com}
%\IEEEauthorblockA{\IEEEauthorrefmark{3}Starfleet Academy, San Francisco, California 96678-2391\\
%Telephone: (800) 555--1212, Fax: (888) 555--1212}
%\IEEEauthorblockA{\IEEEauthorrefmark{4}Tyrell Inc., 123 Replicant Street, Los Angeles, California 90210--4321}}




% use for special paper notices
%\IEEEspecialpapernotice{(Invited Paper)}




% make the title area
\maketitle

% As a general rule, do not put math, special symbols or citations
% in the abstract
% \begin{abstract}
%   Equirectangular images are captures in 360 degrees of user surroundings. Virtual reality applications provide immersive experience when using this type of media. However, in order to develop a 360 viewer it is necessary to choose among diferent media formats, resolution configurations and texture-to-objects mappings. This work proposes to develop a tool integrated into \textit{Unity} editor to automatize the quality assessment of different settings of 360 image visualization. Using objetive metrics, we compare the UV mapping of a procedural sphere, a \textit{Skybox} rendering and a \textit{Cubemap}.
% \end{abstract}

\begin{abstract}
  Imagens equiretangulares são uma captura em 360 graus do entorno do usuário. Aplicações de realidade virtual possibilitam uma experiência imersiva na utilização deste tipo de mídia. No entanto para desenvolver um visualizador de mídia 360 é necessário escolher entre diferentes formatos, configurações de resolução, e formas de mapeamentos para objetos 3D. Este artigo tem a proposta de desenvolver uma ferramenta integrada ao editor \textit{Unity} que permita automatizar a avaliação de qualidade de diferentes parâmetros de visualização de imagens 360. Métricas objetivas foram empregadas na comparação entre os mapeamentos de UV com base em uma esfera procedural, uma exibição de \textit{Skybox} e um \textit{Cubemap}.
\end{abstract}

% no keywords

% For peer review papers, you can put extra information on the cover
% page as needed:
% \ifCLASSOPTIONpeerreview
% \begin{center} \bfseries EDICS Category: 3-BBND \end{center}
% \fi
%
% For peerreview papers, this IEEEtran command inserts a page break and
% creates the second title. It will be ignored for other modes.
\IEEEpeerreviewmaketitle



\section{Introduction}

% TOREVIEW: Contextualização - Imagens 360
%   - O que são imagens 360
%   --- Possibilidade de revisitar um momento do espaço tempo
%   - Câmeras 360
%   - Formatos
%   - Como essas imagens são consumidas
Imagens 360 são uma fotografia completa do entorno, trazendo a ilusão de uma completa visão a partir de um único ponto. Câmeras 360 possuem a funcionalidade de registrar tomadas panorâmicas completas em 360 graus ao redor do ponto de captura. Atualmente, existem diversos formatos possíveis para imagens 360, sendo um dos mais adotados, o formato equiretangular. Capturas 360 são geralmente compartilhadas em mídias sociais, onde através de movimentos gestuais é possível navegar na imagem. Aplicações de Realidade Virtual trazem a possibilidade de uma visão mais imersiva de imagens 360, permitindo ao usuário a sensação de estar dentro da imagem.

% TOREVIEW: Realidade Virtual e displays
Recentemente, dispositivos de realidade virtual (VR, sigla inglês para \textit{Virtual Reality}) se tornaram acessíveis para uma grande parte da população. Neste tipo de aparelho, um mundo virtual é renderizado de maneira estéreo, ou seja, com uma visão para cada olho, passando assim a sensação de profundidade e maior presença em um ambiente virtual. A tecnologia por trás dos displays \textit{VR} tem evoluido muito, porém esbarram na necessidade de oferecer uma grande densidade de pixels por grau de campo de visão. Segundo \cite{va1965visual}, o olho humano possui uma resolução estimada de 60 pixels por grau, ou seja, um aparelho modesto de 100 graus de campo de visão deveria prover uma resolução de 6K para fornecer uma ilusão espacial mais realista possível.

% TODO: Visualizadores de imagens 360
% - Aplicações de visualização simulação a visualização do conteúdo envelopado numa esfera
% - Problemas
% --- Tamanho do formato
% --- Diferentes formatos possuem variados graus de distorção e resolução ao longo da direção visual
Aplicações de visualização de imagens 360 renderizam seu conteúdo envelopado numa esfera, ou seja, como se o conteúdo visual estivesse disposto ao redor do usuário, preenchendo completamente o seu campo visual, como no caso das aplicações \textit{VR}. Para tanto, requerem uma grande quantidade de píxels disponíveis de forma a manter a boa qualidade dessas visualizações. A busca pela melhor qualidade visual pode signicar escolher entre diferentes formatos de exibição, com graus variados de distorção e não uniformidade na resolução ao longo dos 360 graus possíveis.


% Diversos fatores influenciam na qualidade de uma imagem, e a avaliação de pós-processamentos em tais imagens, é um fator primordial para que não reflita na experiência do usuário, principalmente em aplicações que utilizam dispositivos de Realidade Virtual.


% Qualidade de imagem 360
% - Parâmetros de configuração
% - Necessidade de uma ferramenta para comparar diferentes sets de configuração
Ao escolher um formato e resolução para uma imagem 360 é necessário levar em conta qual dispositivo essa será exibida. Assim acaba sendo mandatório, um contínuo ciclo de ajuste e teste no dispositivo até que seja possível encontrar as configurações ideais. O que gera assim a necessidade de uma ferramenta que possa simular e comparar diferentes configurações de imagem e de dispositivo afim de permitir uma escolha mais bem informada.

Existem duas maneiras de avaliação da qualidade de imagem: Métricas Subjetivas e Métricas Objetivas. A primeira, é a realização de testes em que observadores avaliam uma sequência de imagens e a segunda, é uma avaliação automática por meio de algoritmos implementados com base em modelos matemáticos

% TODO: Unity como ferramenta default para desenvolver VR apps
A \textit{Unity} é uma \textit{engine} de desenvolvimento de jogos para plataformas móveis, consoles, computadores, além de dispositivos de realidade virtual e aumentada. Atualmente, a \textit{Unity} é muito utilizada por grupos de desenvolvedores independentes e grandes corporações, tais como Microsoft e Disney, e é a ferramenta mais adotada no desenvolvimento de aplicações \textit{VR}.

% TODO: Proposta
Neste artigo propomos uma ferramenta de avaliação de qualidade de imagem 360 para o editor da \textit{Unity} utilizando métricas objetivas. Para avaliações em condições próximas a de cenários reais, nossa ferramenta deve ser capaz de simular a visualização sob pré-determinados valores de ângulos de campo de visão e resolução.

% TOREPHRASE: Estrutura do artigo
Na seção \ref{sec:relatedworks} analisamos alguns trabalhos que descrevem a relação de compromisso entre formato e a qualidade de imagens 360 e que avaliam imagens 360 utilizando métricas objetivas. Os diferentes formatos e projeções utilizadas foram descritas na seção \ref{sec:imageprojection}. As métricas objetivas adotadas por este trabalho foram compiladas na seção \ref{sec:metrics}. Na seção \ref{sec:unitytool} é feita uma descrição detalhada da nossa implementação. Resultados são discutidos na seção \ref{sec:results}. Por fim, as conclusões e perspectivas de trabalhos futuros são mencionadas na seção \ref{sec:conclusion}.


% An example of a floating figure using the graphicx package.
% Note that \label must occur AFTER (or within) \caption.
% For figures, \caption should occur after the \includegraphics.
% Note that IEEEtran v1.7 and later has special internal code that
% is designed to preserve the operation of \label within \caption
% even when the captionsoff option is in effect. However, because
% of issues like this, it may be the safest practice to put all your
% \label just after \caption rather than within \caption{}.
%
% Reminder: the "draftcls" or "draftclsnofoot", not "draft", class
% option should be used if it is desired that the figures are to be
% displayed while in draft mode.
%
%\begin{figure}[!t]
%\centering
%\includegraphics[width=2.5in]{myfigure}
% where an .eps filename suffix will be assumed under latex,
% and a .pdf suffix will be assumed for pdflatex; or what has been declared
% via \DeclareGraphicsExtensions.
%\caption{Simulation results for the network.}
%\label{fig_sim}
%\end{figure}

% Note that the IEEE typically puts floats only at the top, even when this
% results in a large percentage of a column being occupied by floats.


% An example of a double column floating figure using two subfigures.
% (The subfig.sty package must be loaded for this to work.)
% The subfigure \label commands are set within each subfloat command,
% and the \label for the overall figure must come after \caption.
% \hfil is used as a separator to get equal spacing.
% Watch out that the combined width of all the subfigures on a
% line do not exceed the text width or a line break will occur.
%
%\begin{figure*}[!t]
%\centering
%\subfloat[Case I]{\includegraphics[width=2.5in]{box}%
%\label{fig_first_case}}
%\hfil
%\subfloat[Case II]{\includegraphics[width=2.5in]{box}%
%\label{fig_second_case}}
%\caption{Simulation results for the network.}
%\label{fig_sim}
%\end{figure*}
%
% Note that often IEEE papers with subfigures do not employ subfigure
% captions (using the optional argument to \subfloat[]), but instead will
% reference/describe all of them (a), (b), etc., within the main caption.
% Be aware that for subfig.sty to generate the (a), (b), etc., subfigure
% labels, the optional argument to \subfloat must be present. If a
% subcaption is not desired, just leave its contents blank,
% e.g., \subfloat[].


% An example of a floating table. Note that, for IEEE style tables, the
% \caption command should come BEFORE the table and, given that table
% captions serve much like titles, are usually capitalized except for words
% such as a, an, and, as, at, but, by, for, in, nor, of, on, or, the, to
% and up, which are usually not capitalized unless they are the first or
% last word of the caption. Table text will default to \footnotesize as
% the IEEE normally uses this smaller font for tables.
% The \label must come after \caption as always.
%
%\begin{table}[!t]
%% increase table row spacing, adjust to taste
%\renewcommand{\arraystretch}{1.3}
% if using array.sty, it might be a good idea to tweak the value of
% \extrarowheight as needed to properly center the text within the cells
%\caption{An Example of a Table}
%\label{table_example}
%\centering
%% Some packages, such as MDW tools, offer better commands for making tables
%% than the plain LaTeX2e tabular which is used here.
%\begin{tabular}{|c||c|}
%\hline
%One & Two\\
%\hline
%Three & Four\\
%\hline
%\end{tabular}
%\end{table}


% Note that the IEEE does not put floats in the very first column
% - or typically anywhere on the first page for that matter. Also,
% in-text middle ("here") positioning is typically not used, but it
% is allowed and encouraged for Computer Society conferences (but
% not Computer Society journals). Most IEEE journals/conferences use
% top floats exclusively.
% Note that, LaTeX2e, unlike IEEE journals/conferences, places
% footnotes above bottom floats. This can be corrected via the
% \fnbelowfloat command of the stfloats package.

\section{Trabalhos Relacionados} \label{sec:relatedworks}

% TODO: Criar Aplicações em VR é sobre como preencher o espaço em 3D. Exemplos.
Aplicações VR diferenciam-se de outras aplicações pela preocupação em preencher o espaço tridimensional de forma a fornecer conteúdo para todos os pontos de vista possíveis. Segundo \cite{fuchs2017virtual} uma importante propriedade dos headsets VR é a habilidade de isolar visualmente e acusticamente o espectador do mundo real. Assim o sujeito pode estar completamente imerso em 360 graus de ambiente virtual

% TODO: Ferramentas Unity

% TODO: Imagens 360
Segundo o trabalho \cite{sacht2010face}, identificar rostos e linhas retas em uma imagem equirectangular requer utilizar as projeções de perspectiva e de Mercator. Ao projetar a imagem através de Mercator, as formas são preservadas e a identificação de rosto através de um detector de faces básico é possível


O trabalho de \cite{dunn2017resolution} ressalta bem o quanto o formato de uma imagem 360 impactam em sua resolução e a uniformidade. Em imagens equiretangulares ocorre uma alta resolução nos pólos e uma alta uniformidade no meio horizontal. Enquanto em um \textit{cubemap} ocorre uma alta resolução nos cantos e bordas além de uma alta uniformidade nas diagonais de cada face.

% TODO: métricas que avaliam qualidade de imagem
% - Tipos de métricas
% --- Métricas subjetivas
% --- Métricas objetivas
A melhor metodologia para avaliar a qualidade de imagens é por meio de Avaliações Subjetivas. Em tais testes, observadores humanos são convidados a visualizar uma série de imagens e avalia-las. Uma referência comum utilizada para tais avaliações é o MOS (Mean Opinion Score), onde cada avaliador dar uma nota em uma escala de 1(péssimo) a 5 (excelente), que representa a percepção da qualidade de imagem esperada pelo usuário. O MOS é gerado pelo resultado da média do conjunto das avaliações subjetiva para cada amostra.

 Existem diversas metodologias para avaliação subjetivas tai como: Single Stimulus Continuous Quality Scale (SSCQS), em que uma imagem referência é apresentada seguida de uma longa sequência de amostras a serem avaliadas ao longo do tempo. Outra metologia é Subjective Assessment of Multimedia Video Quality (SAMVIQ), que consiste em comparar uma imagem referência e uma imagem processada, para cada amostra de uma sequência. Afim de aumentar a eficiente deste último, foi proposto em \cite{zhang2017subjective} a avaliação subjetiva de qualidade de vídeo de mídia panorâmica (SAMPVIQ - Subjective Assessment of Multimedia Panoramic Video Quality), que é uma melhoria com base em vídeos panorâmicos. Segundo \cite{zhang2017subjective}, SAMPVIQ apresentou uma eficiência superior aos anteriores por apresentar resultados mais compatíveis com métricas objetivas presentes na literatura.

Embora os testes subjetivos sejam geralmente precisos se realizados corretamente, eles são inconvenientes, caros e demorados. Principalmente, quando aplicados em avaliação de imagens e vídeos para Realidade Virtual onde a experiência do usuário é mais imersiva. Assim, uma medida de desempenho objetiva que pudesse prever com precisão a percepção humana seria um valioso método complementar.

% TODO: Trabalhos que avaliam métricas em videos/imagens 360


\section{Projeção de Imagens 360 como Mapeamento de UV}  \label{sec:imageprojection}

Exibir imagens 360 significa preencher completamente o campo de visão do usuário, ou seja, prover conteúdo visual em 360 graus. Considerando o formato de imagem equiretangular, algumas possibilidades são listadas abaixo:

\begin{enumerate}
  \begin{item}Utilizar um \textit{Mesh} em formato de esfera para exibir a imagem equiretangular em seu interior\end{item}
  \begin{item}Utilizar o \textit{Skybox} para renderizar a imagem 360 como pano de fundo.\end{item}
  \begin{item}Mapear a imagem 360 em posições de UV de um \textit{Mesh} em formato cúbico.\end{item}
\end{enumerate}

Cada formato, possui suas vantagens e desvantagens em termos de resolução oferecida por direção angular e geração de distorção ao longo da visualização.

% Estrutura das subseções
Nas subseções seguintes serão detalhados os cálculos necessários para a implementação do mapeamento equiretangular na esfera (subseção \ref{subsec:equimap}), visualização da imagem 360 em um skybox (subseção \ref{subsec:skyboxviz}) e mapeamento \textit{Cubemap} de uma image equiretangular (subseção \ref{subsec:equicubemap}).


\subsection{Mapeamento de Imagens Equiretangulares em uma Esfera} \label{subsec:equimap}

O formato de imagem equiretangular é uma captura ou produção de imagem feita para ser exibida no interior de uma esfera possibilitando preencher completamente o campo de visão do usuário, além de ser a solução muito utilizada na exibição de imagens equiretangulares em 360 graus. O mapeamento de UV da esfera é o padrão adotado para \textit{softwares} de modelagem na geração de uma esfera: geração de vertíces baseada em coordenadas de longitude/latitude.

\begin{figure}[!tbp]
  \centering
  \begin{minipage}[b]{0.40\textwidth}
    \includegraphics[width=1.4\textwidth]{../images/equirect_projection}
    \caption{Imagem 360 em formato equiretangular.}
    \label{fig:equirectimage}
  \end{minipage}
  \hfill
  \begin{minipage}[b]{0.44\textwidth}
    \centering
    \includegraphics[width=0.7\textwidth]{../images/sphere.png}
    \caption{Mapeamento de UV de uma imagem equiretangular para uma esfera.}
    \label{fig:equisphere}
  \end{minipage}
\end{figure}

% 2 - Mapeamento de UV em uma esfera invertida
Na figura \ref{fig:sphere_transversalsec} é exibida uma secção transversal da divisão de uma esfera em longitudes. Na geração das posições dos vértices da esfera, se faz necessário definir uma quantidade de valores de longitude $N$, assim o tamanho angular $T$ para divisão longitudinal pode ser calculado pela equação \ref{longitudesize}.

\begin{figure}[ht]
\centering
\includegraphics[width=.25\textwidth]{../images/longitudes.png}
\caption{Uma secção transversal da esfera dividida em 8 longitudes.}
\label{fig:sphere_transversalsec}
\end{figure}

\begin{equation}
T = \frac{2 \pi}{N}
\label{longitudesize}
\end{equation}

O tamanho angular total de uma quantidade de $i$ de valores de longitude pode ser dada pela equação \ref{longitudealpha}.

\begin{equation}
\alpha_{i} = i * T
\label{longitudealpha}
\end{equation}

Considerando a figura \ref{fig:sphere_transversalsec}, percebe-se que o seno e cosseno do ângulo T definem-se as posições X e Z dos pontos da esfera pertencentes a essa secção transversal da esfera. Dessa forma, supondo uma esfera de raio $R$, podemos escrever as equações \ref{x_d} e \ref{z_d}.

\begin{equation}
x_{i} = R * \sin(\alpha_{i})
\label{x_d}
\end{equation}

\begin{equation}
z_{i} = R * \cos(\alpha_{i})
\label{z_d}
\end{equation}

Em um corte longitudinal, é possível perceber que o raio $R$ de uma secção transversal varia ao longo da altura da esfera. A figura \ref{fig:sphere_longitudisec} mostra a secção longitudinal de uma esfera dividida em algumas latitudes. Determina-se então um valor $K$ como o tamanho angular de um valor de latitude da esfera, dado um valor $M$ de latitudes, tal como visto na equação \ref{equation5}.

\begin{figure}[ht]
\centering
\includegraphics[width=.25\textwidth]{../images/latitudes.png}
\caption{Uma secção longitudinal da esfera dividida em 8 latitudes.}
\label{fig:sphere_longitudisec}
\end{figure}

\begin{equation}
K = \frac{\pi}{M}
\label{equation5}
\end{equation}

O tamanho angular total de uma quantidade de $i$ de valores de latitude pode ser dado pela equação \ref{equation6}.

\begin{equation}
\alpha_{yi} = i * K
\label{equation6}
\end{equation}

A posição Y dos pontos da esfera, considerando raio unitário, pode ser dada pela equação \ref{equation7}.
\begin{equation}
y_{i} = \cos(\alpha_{yi})
\label{equation7}
\end{equation}

O raio $D_{yi}$ obtido em uma secção transversal na latitude $i$ é definido na equação \ref{equation8} como:
\begin{equation}
R_{yi} = \sin(\alpha_{yi})
\label{equation8}
\end{equation}

Aplicando-se a equação \ref{equation8} nas equações \ref{x_d} e \ref{z_d} obtém-se as posições X e Z dos vértices da esfera em função de suas coordenadas de longitude e latitude.

\begin{equation}
x_{i} = \sin(\alpha_{yi}) * \sin(\alpha_i)
\label{equation9}
\end{equation}

\begin{equation}
z_{i} = \sin(\alpha_{yi}) * \cos(\alpha_i)
\label{equation10}
\end{equation}

\subsection{Visualização em um Skybox} \label{subsec:skyboxviz}

% Definir Skybox
Um \textit{Skybox} define o conteúdo a ser renderizado no fundo de uma visualização gráfica de um ambiente virtual em 3D.

% Como mapear a direção de visualização para posições de UV
% TODO: Explicar o uso do termo fragmento
No processo de rasterização do conteúdo de fundo, é necessário obter uma coordenada de UV para cada pixel (ou fragmento) exibido na tela. Normalmente implementações de \textit{shaders} de \textit{Skybox} utilizam texturas 3D para armazenar as 6 faces de um cubo e as acessa por meio de uma chamada da função gráfica \textit{tex3D}.

% Mapeamento de uma esfera contínua
Assim, enquanto a abordagem discutida na subseção \ref{subsec:equimap} define uma projeção de coordenadas de UVs para posições no espaço 3D, a abordagem utilizando \textit{Skybox} segue um método inverso: a partir de posições normalizadas no espaço 3D busca encontrar coordenadas de UV equivalentes de maneira contínua em termos de possibilidade de direção de visualização.

\subsection{Mapeamento Cubemap a partir de Imagens Equiretangulares} \label{subsec:equicubemap}

% Geração de um cubo.


% Calculo de UV


% Percebe-se que mapeamento equivale a discritização da abordagem anterior
Equivale a discretização da abordagem de mapeamento contínuo das coordenadas de UV em um \textit{Skybox}.


% Necessidade de subdivisão de mesh


% Subdivisão de mesh, um trinagule em tres tringulos


% Subdivisão de mesh, um trinagule em quatro tringulos


\section{Métricas de Qualidades de Imagens 360}  \label{sec:metrics}

%  Métricas objetivas
O objetivo da avaliação objetiva é desenvolver uma medida quantitativa que pode prever a qualidade da imagem percebida. É difícil encontrar uma única medida de avaliação objetiva, fácil de calcular, que corresponda bem com a inspeção visual e seja adequado para uma variedade de exigências da aplicação. É necessário a avaliação da qualidade por meio de várias medidas objetivas.


% TODO: Listar métricas utilizadas no artigo

% TODO: Subsection para cada métrica. Exemplificar com screenshots.

\subsection{MSE}

MSE é uma sigla para "mínimo erro quadrático" ou "Mean Square Error" em inglês. A equação \ref{eq:mse} descreve como é calculado:
\begin{equation}
MSE=\frac{1}{MN}\sum_{m=0}^{M-1}{\sum_{n=0}^{N-1}{e(m,n)^2}}
\label{eq:mse}
\end{equation}

Onde, o $e$ indica o erro da diferença nos valores de pixels das duas imagens. $M$ e $N$ indicam a largura e altura das imagens, respectivamente, e $m$ e $n$ indicam a posição horizontal e vertical, respectivamente.

\subsection{SSIM}

Descrita em \cite{wang2004image}, a métrica SSIM pode ser calculada pela equação \ref{eq:ssim}.

\begin{equation}
SSIM(x,y)=\frac{(2*\mu_x*\mu_y+C_1)*(2*\sigma_{xy}+C_2)}{(\mu^2_x+\mu^2_y+C_1)*(\sigma^2_x+\sigma^2_y+C_2)}
\label{eq:ssim}
\end{equation}

\subsection{PSNR}

Peak Signal-to-noise ratio (PSNR) É a medida mais utilizada para avaliação de qualidade é calculada:

\begin{equation}
PSNR = 10*log_{10}{\frac{(2^n-1)^2}{MSE}}
\label{eq:psnr}
\end{equation}

Aqui $n$ representa o \textit{bit depth} de um pixel.

Segundo \cite{yu2015framework}, uma desvantagem da aplicação do PSNR é que todos os pixels possuem pesos iguais, porém em uma imagem no formato equiretangular existe uma alta redundância espacial nas áreas superiores e inferiores. Como em nossa ferramenta a comparação se dá na visualização final da câmera, essa desvantagem não nos afeta muito, pois o resultado final não é em formato equiretangular, apesar de ter sido derivado de um.

% \subsection{WS-PSNR}

% Na métrica PSNR todos os pixels têm pesos iguais, tal como descrito em \cite{yu2015framework}. Entretanto, em uma imagem no formato equiretangular existe uma alta redundância espacial nas áreas superiores e inferiores. Assim, Weighted Spherical PSNR (WS-PSNR), foi proposto para formatos equiretangular, onde pesos diferentes são atribuídos dependendo da latitude:

% \begin{equation}
% Weight=\frac{cos(j - (\frac{height}{2} - 0,5))*3,141592}{height}
% \label{eq:wspsnr}
% \end{equation}

% Onde, j indica a latitude na altura da imagem. Quando j está perto de (height / 2), o peso é próximo de 1. Por outro lado, o peso torna-se pequeno quando j está mais próximo das áreas polares.

\section{Implementação de Ferramenta no Editor Unity}  \label{sec:unitytool}

% implementação - Uso
A ferramenta proposta permite definir configuração de screenshots de uma imagem 360 exibida no espaço 3D e variar configurações de imagem ou de projeto para traçar comparações de qualidade de imagem utilizando métricas objetivas definidas na seção \ref{sec:metrics}.

% Implementação - visual alto nível
A arquitetura da nossa implementação envolve uma camada de configuração em C\# na Unity e uma camada em python para o cálculo das métricas nas imagens geradas pela \textit{Unity}. A comunicação entre camadas ocorre por meio da criação de novos processos dentro do editor \textit{Unity}. A figura \ref{fig:toolarch} exibe como os componentes estão conectados na arquitetura da ferramenta.

\begin{figure}[ht]
\centering
\includegraphics[width=\textwidth]{../images/tool_arch.png}
\caption{Arquitetura da ferramenta proposta.}
\label{fig:toolarch}
\end{figure}

% Descrever camada Unity
Para permitir um uso rápido e mais amigável, foi desenvolvida uma interface de editor, com um \textit{Inspector} customizado, ou seja, uma visão customizada do nosso componente em C\#, conforme vista na figura \ref{fig:unity_tool}. Atráves desse componente é possível definir qual o ângulo do campo de visão, a largura e altura do \textit{screenshot} a ser gerado, as direções de \textit{screenshot}, os objetos de visualização a serem utilizados, definir as métricas de comparação e definir se serão gerados gráficos ou um relatório ao final do processo.

\begin{figure}[ht]
\centering
\includegraphics[width=0.4\textwidth]{../images/tool.png}
\caption{Ferramenta de editor na Unity}
\label{fig:unity_tool}
\end{figure}

% Descrever camada Python
A camada \textit{python} ficou responsável pela avaliação de pares de imagens passadas via processo pela \textit{Unity}. Utilizando as bibliotecas \textit{scikit}, \textit{numpy} e \textit{matplotlib}, as imagens são avaliadas e o resultado de cada métrica é salvo em um relatório ao final, sumarizando todos os resultados. Assim, basta ao usuário da ferramenta abrir o relatório e identificar qual forma de visualização obteve a melhor avaliação.

\section{Resultados}  \label{sec:results}

% Ferramenta implementada
Utilizamos a \textit{Unity 2017.3.1f} e \textit{python 2.7} para realizar a implementação da ferramenta proposta. A ferramenta pode ser incorporada em qualquer projeto \textit{Unity} por meio da importação de um \textit{unitypackage}, um formato padrão da \textit{Unity} para componentização de recursos e ferramentas.

% Comparacao de resultados
\begin{tabular}{l*{3}{c}r}
Métricas          & MSE & SSIM & PSNR \\
\hline
Direção 0 - Cubemap & 176,22 & 0,93 & 25,67 \\
Direção 1 - Cubemap & 125,63 & 0,93 & 27,14 \\
Direção 2 - Cubemap & 13,83 & 0,97 & 36,72 \\
Direção 0 - Esfera  & 88,56 & 0,88 & 28,66 \\
Direção 1 - Esfera  & 242,47 & 0,76 & 24,28 \\
Direção 2 - Esfera  & 96,26 & 0,86 & 28,30 \\
\label{tab:metrics_results}
\end{tabular}

% TODO: Comentar resultados obtidos com as métricas
Conforme a tabela \ref{tab:metrics_results}, na direção 0, o mapeamento da esfera se sobressaiu em relação ao \textit{Cubemap} devido a um erro na interpolação dos valores entre 0.9 e 0.1.

\begin{figure}[!tbp]
  \centering
  \begin{minipage}[b]{0.3\textwidth}
    \includegraphics[width=1.1\textwidth]{../images/screenshots/Screenshot_0_Equi2Cube}
    \caption{Cubemap na direção 0}
    \label{fig:sphere_direction_0}
  \end{minipage}
  \hfill
  \begin{minipage}[b]{0.3\textwidth}
    \centering
    \includegraphics[width=1.1\textwidth]{../images/screenshots/Screenshot_0_Skybox}
    \caption{Skybox na direção 0}
    \label{fig:equisphere}
  \end{minipage}
  \hfill
  \begin{minipage}[b]{0.3\textwidth}
    \centering
    \includegraphics[width=1.1\textwidth]{../images/screenshots/Screenshot_0_Sphere}
    \caption{Esfera na direção 0}
    \label{fig:equisphere}
  \end{minipage}
\end{figure}

\begin{figure}[!tbp]
  \centering
  \begin{minipage}[b]{0.3\textwidth}
    \includegraphics[width=1.1\textwidth]{../images/screenshots/Screenshot_2_Equi2Cube}
    \caption{Cubemap na direção 1}
    \label{fig:sphere_direction_0}
  \end{minipage}
  \hfill
  \begin{minipage}[b]{0.3\textwidth}
    \centering
    \includegraphics[width=1.1\textwidth]{../images/screenshots/Screenshot_2_Skybox}
    \caption{Skybox na direção 1}
    \label{fig:equisphere}
  \end{minipage}
  \hfill
  \begin{minipage}[b]{0.3\textwidth}
    \centering
    \includegraphics[width=1.1\textwidth]{../images/screenshots/Screenshot_2_Sphere}
    \caption{Esfera na direção 1}
    \label{fig:equisphere}
  \end{minipage}
\end{figure}

\begin{figure}[!tbp]
  \centering
  \begin{minipage}[b]{0.3\textwidth}
    \includegraphics[width=1.1\textwidth]{../images/screenshots/Screenshot_5_Equi2Cube}
    \caption{Cubemap na direção 2}
    \label{fig:sphere_direction_0}
  \end{minipage}
  \hfill
  \begin{minipage}[b]{0.3\textwidth}
    \centering
    \includegraphics[width=1.1\textwidth]{../images/screenshots/Screenshot_5_Skybox}
    \caption{Skybox na direção 2}
    \label{fig:equisphere}
  \end{minipage}
  \hfill
  \begin{minipage}[b]{0.3\textwidth}
    \centering
    \includegraphics[width=1.1\textwidth]{../images/screenshots/Screenshot_5_Sphere}
    \caption{Esfera na direção 2}
    \label{fig:equisphere}
  \end{minipage}
\end{figure}

\section{Conclusão}  \label{sec:conclusion}

% TODO: Sintetizar o desenvolvimento da proposta
Neste artigo propomos o desenvolvimento de uma ferramenta para avaliação de qualidade de imagens 360 utilizando métricas objetivas.

% TODO: Qual o valor de uma ferramenta de qualidade de imagens para o editor da Unity
Nossa implementação foi destinada para uso de dentro do editor \textit{Unity}, vista essa ser uma ferramenta bastante popularizada para o desenvolvimento de aplicações de realidade virtual.

Utilizamos nossa ferramenta para comparar a qualidade de imagens 360 geradas por diferentes mapeamento de UV: latitudes e longitudes em uma esfera invertida, exibição em um Skybox e conversão para cubemap.

% TODO: Trabalhos futuros
Uma das desvantagens da nossa ferramenta é a necessidade do usuário conhecer a fundo as métricas utilizadas para ser capaz de tomar a melhor decisão de visualização 360. Em trabalhos futuros, planejamos utilizar o cálculo das métricas para definir um valor de avaliação único, indicando de forma automatizada o melhor resultado. Outro ponto identificado foi que a visualização nos dispositivos móveis é mais precisa quanto a visualização final, por esse motivo planejamos também um componente embarcado na aplicação que permite colher resultados durante a execução da aplicação no dispositivo móvel.

\section{Conclusion}
The conclusion goes here.




% conference papers do not normally have an appendix


% use section* for acknowledgment
\section*{Acknowledgment}


The authors would like to thank...





% trigger a \newpage just before the given reference
% number - used to balance the columns on the last page
% adjust value as needed - may need to be readjusted if
% the document is modified later
%\IEEEtriggeratref{8}
% The "triggered" command can be changed if desired:
%\IEEEtriggercmd{\enlargethispage{-5in}}

% references section

% can use a bibliography generated by BibTeX as a .bbl file
% BibTeX documentation can be easily obtained at:
% http://mirror.ctan.org/biblio/bibtex/contrib/doc/
% The IEEEtran BibTeX style support page is at:
% http://www.michaelshell.org/tex/ieeetran/bibtex/
\bibliographystyle{IEEEtran}

% \bibliographystyle{ieeetran}
\bibliography{bare_conf}
% argument is your BibTeX string definitions and bibliography database(s)
% \bibliography{IEEEabrv,qualityrefs}
%
% <OR> manually copy in the resultant .bbl file
% set second argument of \begin to the number of references
% (used to reserve space for the reference number labels box)
% \begin{thebibliography}{1}

% \bibitem{IEEEhowto:kopka}
% H.~Kopka and P.~W. Daly, \emph{A Guide to \LaTeX}, 3rd~ed.\hskip 1em plus
%   0.5em minus 0.4em\relax Harlow, England: Addison-Wesley, 1999.

% \end{thebibliography}

% that's all folks
\end{document}


